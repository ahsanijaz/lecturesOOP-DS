\documentclass{beamer}
\setbeamercovered{transparent}
\usepackage{epstopdf}
\usepackage{listings}
\usepackage{lipsum}
\usepackage{subfig}
\usepackage{algorithm}
\usepackage{algorithmicx}
\usepackage{cite}
\usepackage{lipsum}
\usepackage{amssymb}
\usepackage{color}
\usepackage{IEEEtrantools}
\usepackage{booktabs}
\usepackage{texpower}
\usepackage{amsmath}
\usepackage{caption}
\usepackage{multirow}
\usepackage{graphicx}
\newtheorem{Key points}{Key points}
\newtheorem{Summary}{Summary}
\usepackage{dblfloatfix}
%\usepackage{adjustbox}
%\usepackage{animate}
%\usepackage{movie15}
%\usepackage{subfig}
%\newtheorem{Definition}{Definition}
%\usepackage[font={small}]{caption}
\usepackage{beamerthemeshadow}
\newcommand\Fontvi{\fontsize{5}{6.2}\selectfont}
\newcommand\Fontvia{\fontsize{6}{7.2}\selectfont}
\newcommand\Fontviaa{\fontsize{8}{7.2}\selectfont}
\usepackage{listings}
\lstset{language=C++,
                keywordstyle=\color{blue},
                stringstyle=\color{red},
                commentstyle=\color{green},
                morecomment=[l][\color{magenta}]{\#},
                numbers=left,
                escapeinside=||
}

%\captionsetup{font=scriptsize,labelfont=scriptsize}
 \usetheme{Antibes}%PaloAlto
\begin{document}
\title[Lecture 3]{Data Structures and Object Oriented Programming using C++} 
\author[]{Ahsan Ijaz}
\date{September 16, 2013}
 \frame{\titlepage}
% \AtBeginSection[]
% {
% \begin{frame}<beamer>{Table of Contents}
% \tableofcontents[currentsection,currentsubsection, 
%     hideothersubsections, 
%     sectionstyle=show/shaded,
% ]
% \end{frame}
% }
\section{Review of C++}
\frame{\frametitle{Topics Covered}
  \begin{itemize}
  \item<1-> Basic Data Types
  \item<2-> Variables, Constants 
  \item<3-> Functions, Function Overloading
  \item<4-> Pointers
  \item<5-> Arrays
  \item<6-> Dynamic Memory Allocation
  \end{itemize}
}

\begin{frame}[fragile]
\frametitle{Basic Program}
%\Fontvia
%\begin{small}
\begin{lstlisting}

#include <iostream>
using namespace std;
/*Loads of Comments*/
int main ()
{
  cout << "Hello World!"; // But of course! 
  return 0;
}
\end{lstlisting}
\end{frame}

\subsection{Data Types}
\frame{\frametitle{Data Types}
\begin{table}[h]%[hbt]
  \centering
  \begin{tabular}{llc}\toprule
    Name & Description & Size~(byte)\\ \midrule
    char & Character & 1 \\
    short int~(short) & Short integer & 2\\
    int & Integer  & 4\\
    long int~(long) & Long integer & 4 \\
    bool & Boolean value (1 or 0) & 1\\
    float & Floating point variable & 4\\
    double & Double precision floating point & 8 \\  \bottomrule
  \end{tabular}
 \caption{Data Types}
\end{table}
}
\begin{frame}[fragile]
\frametitle{Variables Example}
\Fontviaa
%\begin{small}
\begin{lstlisting}
// operating with variables

#include <iostream>
using namespace std;

int main ()
{
  // declaring variables:
  int a
  float b;
  int result;

  // process:
  a = 5;
  b = 2.3;
  result = a + b;

  // print out the result:
  cout << result;

  // terminate the program:
  return 0;
}
\end{lstlisting}
\end{frame}

\subsection{Control Sequences}
\begin{frame}[fragile]
\frametitle{if else sequence}
%\Fontviaa
%\begin{small}
\begin{lstlisting}
if (x == 100)
{
   cout << "x is ";
   cout << x;
}
else if (x < 0)
  cout << "x is negative";
else
  cout << "x is 0";
\end{lstlisting}
\end{frame}
\begin{frame}[fragile]
\frametitle{While loop}
%\Fontviaa
%\begin{small}
\begin{lstlisting}
// custom countdown using while
#include <iostream>
using namespace std;
int main ()
{
  int n;
  cout << "Enter the starting number > ";
  cin >> n;
  while (n>0) {
    cout << n << ", ";
    --n;
  }
  cout << "FIRE!\n";
  return 0;
}
\end{lstlisting}
\end{frame}
\begin{frame}[fragile]
\frametitle{for loop}
%\Fontviaa
%\begin{small}
\begin{lstlisting}

for ( n=0, i=100 ; n!=i ; n++, i-- )
{
   // whatever here...
}
\end{lstlisting}
\end{frame}
\begin{frame}[fragile]
\frametitle{break statement}
\begin{columns}[onlytextwidth]
    \begin{column}{0.6\textwidth}
      \centering
\Fontviaa
%\begin{small}
\begin{lstlisting}
// break loop example
#include <iostream>
using namespace std;
int main ()
{
  int n;
  for (n=10; n>0; n--)
  {
    cout << n << ", ";
    if (n==3)
    {
      cout << "countdown aborted!";
      break;
    }
  }
  return 0;
}
\end{lstlisting}
    \end{column}
\pause    \begin{column}{0.3\textwidth}
      \centering
\textbf{{\color{blue}Output:}} 10, 9, 8, 7, 6, 5, 4, 3, countdown aborted!
    \end{column}
\end{columns}
\end{frame}
\begin{frame}[fragile]
\frametitle{continue statement}
\begin{columns}[onlytextwidth]
    \begin{column}{0.6\textwidth}
      \centering
\Fontviaa
%\begin{small}
\begin{lstlisting}
// continue loop example
#include <iostream>
using namespace std;

int main ()
{
  for (int n=10; n>0; n--) {
    if (n==5) continue;
    cout << n;
  }
  cout << "FIRE!\n";
  return 0;
}
\end{lstlisting}
    \end{column}
\pause    \begin{column}{0.4\textwidth}
      \centering
\textbf{{\color{blue}Output:}} 10, 9, 8, 7, 6, 4, 3, 2, 1, FIRE!
    \end{column}
\end{columns}
\end{frame}
\begin{frame}[fragile]
\frametitle{Switch-case}
\begin{columns}[onlytextwidth]
    \begin{column}{0.5\textwidth}
      \centering
\Fontviaa
%\begin{small}
\begin{lstlisting}
switch (x) {
  case 1:
    cout << "x is 1";
    break;
  case 2:
    cout << "x is 2";
    break;
  default:
    cout << "value of x unknown";
  }
\end{lstlisting}
    \end{column}
    \begin{column}{0.4\textwidth}
      \centering
\Fontviaa
\begin{lstlisting}
if (x == 1) {
  cout << "x is 1";
  }
else if (x == 2) {
  cout << "x is 2";
  }
else {
  cout << "value of x unknown";
  }
\end{lstlisting}
    \end{column}
\end{columns}
\end{frame}

\subsection{Functions}
\begin{frame}[fragile]
\frametitle{Function and Scope of Variables}
\begin{columns}[onlytextwidth]
    \begin{column}{0.6\textwidth}
      \centering
\Fontvia
%\begin{small}
\begin{lstlisting}
#include <stdio.h>
/* global variable declaration */
int a = 20;
 
int main ()
{
  /* local variable declaration in main function */
  int a = 10;
  int b = 20;
  int c = 0;
  printf ("value of a in main() = %d\n",  a);
  c = sum( a, b);
  printf ("value of c in main() = %d\n",  c);

  return 0;
}
/* function to add two integers */
int sum(int a, int b)
{
    printf ("value of a in sum() = %d\n",  a);
    printf ("value of b in sum() = %d\n",  b);
     { 
     int a = 50;
   printf ("value of a in block = %d\n",  a);
     }
    return a + b;
}
\end{lstlisting}
    \end{column}
\pause    \begin{column}{0.4\textwidth}
      \centering
\textbf{{\color{blue}Output:}}\\ value of a in main() = 10\\
value of a in sum() = 10\\
value of b in sum() = 20\\
value of a in block = 50\\
value of c in main() = 30 
    \end{column}
\end{columns}
\end{frame}
\subsection{Functions}
\begin{frame}[fragile]

\frametitle{Overloaded Functions}
\Fontvia
%\begin{small}
\begin{lstlisting}
// overloaded function
#include <iostream>
using namespace std;
int operate (int a, int b)
{
  return (a*b);
}
float operate (float a, float b)
{
  return (a/b);
}

int main ()
{
  int x=5,y=2;
  float n=5.0,m=2.0;
  cout << operate (x,y);
  cout << "\n";
  cout << operate (n,m);
  cout << "\n";
  return 0;
}
\end{lstlisting}
\textbf{{\color{blue}Output:}}\\10\\2.5

\end{frame}

\subsection{Pointers}
\begin{frame}[fragile]
\frametitle{Basic Example}
\Fontviaa
\begin{lstlisting}
// my first pointer
#include <iostream>
using namespace std;

int main ()
{
  int firstvalue, secondvalue;
  int * mypointer;

  mypointer = &firstvalue;
  *mypointer = 10;
  mypointer = &secondvalue;
  *mypointer = 20;
  cout << "firstvalue is " << firstvalue << endl;
  cout << "secondvalue is " << secondvalue << endl;
  return 0;
}
\end{lstlisting}
\end{frame}
\begin{frame}[fragile]
\frametitle{Pointers Example 2}
\Fontvia
\begin{lstlisting}
// more pointers
#include <iostream>
using namespace std;
int main ()
{
  int firstvalue = 5, secondvalue = 15;
  int * p1, * p2;
  cout << "firstvalue is " << firstvalue << endl;
  cout << "secondvalue is " << secondvalue << endl;
  p1 = &firstvalue;  // p1 = address of firstvalue
  p2 = &secondvalue; // p2 = address of secondvalue
  *p1 = 10;          // value pointed by p1 = 10
  *p2 = *p1;         // value pointed by p2 = value pointed by p1
  cout << "firstvalue is " << firstvalue << endl;
  cout << "secondvalue is " << secondvalue << endl;
  p1 = p2;           // p1 = p2 (value of pointer is copied)
  *p1 = 20;          // value pointed by p1 = 20
  cout << "firstvalue is " << firstvalue << endl;
  cout << "secondvalue is " << secondvalue << endl;
  return 0;
}
\end{lstlisting}
\textbf{{\color{blue}Output:}}
firstvalue is 5\\
secondvalue is 15\\
firstvalue is 10\\
secondvalue is 10\\
firstvalue is 10\\
secondvalue is 20\\
\end{frame}
\begin{frame}[fragile]
\frametitle{Pointers and Arrays}
\begin{columns}[onlytextwidth]
    \begin{column}{0.6\textwidth}
      \centering
\Fontvia
%\begin{small}
\begin{lstlisting}
// more pointers
#include <iostream>
using namespace std;

int main ()
{
  int numbers[5];
  int * p;
  p = numbers;  *p = 10;
  p++;  *p = 20;
  p = &numbers[2];  *p = 30;
  p = numbers + 3;  *p = 40;
  p = numbers;  *(p+4) = 50;
  for (int n=0; n<5; n++)
    cout << numbers[n] << ", ";
  return 0;
}
\end{lstlisting}
    \end{column}
\pause    \begin{column}{0.3\textwidth}
      \centering
\textbf{{\color{blue}Output:}} 10, 20, 30, 40, 50
    \end{column}
\end{columns}
\Fontviaa
\begin{definition}
\centering
array := \&array[0]
\end{definition}
\begin{lstlisting}
a[5] = 0;       // a [offset of 5] = 0
*(a+5) = 0;     // pointed by (a+5) = 0
\end{lstlisting}
\end{frame}
\begin{frame}[fragile]
\frametitle{Pointer to functions}
\Fontvia
%\begin{small}
\begin{lstlisting}
// pointer to functions
#include <iostream>
using namespace std;

int addition (int a, int b)
{ return (a+b); }

int subtraction (int a, int b)
{ return (a-b); }

int operation (int x, int y, int (*functocall)(int,int))
{
  int g;
  g = (*functocall)(x,y);
  return (g);
}

int main ()
{
  int m,n;
  int (*minus)(int,int) = subtraction;

  m = operation (7, 5, addition);
  n = operation (20, m, minus);
  cout <<n;
  return 0;
}
\end{lstlisting}
\end{frame}
\begin{frame}[fragile]
\frametitle{Returning multiple values from functions}
\Fontviaa
%\begin{small}
\begin{lstlisting}
// more than one returning value
#include <iostream>
using namespace std;

void prevnext (int x, int& prev, int& next)
{
  prev = x-1;
  next = x+1;
}
int main ()
{
  int x=100, y, z;
  prevnext (x, y, z);
  cout << "Previous=" << y << ", Next=" << z;
  return 0;
}
\end{lstlisting}
\end{frame}
\begin{frame}[fragile]
\frametitle{Dynamic Memory Allocation}
\Fontvia
%\begin{small}
\begin{lstlisting}
#include <iostream>
using namespace std;
int main ()
{
  int i,n;
  int * p;
  cout << "How many numbers would you like to type? ";
  cin >> i;
  p= new int[i];
  if (p == 0)
    cout << "Error: memory could not be allocated";
  else
  {
    for (n=0; n<i; n++)
    {
      cout << "Enter number: ";
      cin >> p[n];
    }
    cout << "You have entered: ";
    for (n=0; n<i; n++)
      cout << p[n] << ", ";
    delete[] p;
  }
  return 0;
}
\end{lstlisting}
\end{frame}

\end{document}