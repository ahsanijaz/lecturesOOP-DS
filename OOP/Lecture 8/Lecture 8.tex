 \documentclass{beamer}
\setbeamercovered{transparent}
\usepackage{epstopdf}
\usepackage{listings}
\usepackage{lipsum}
\usepackage{subfig}
\usepackage{algorithm}
\usepackage{algorithmicx}
\usepackage{cite}
\usepackage{lipsum}
\usepackage{amssymb}
\usepackage{color}
\usepackage{IEEEtrantools}
\usepackage{booktabs}
\usepackage{texpower}
\usepackage{amsmath}
\usepackage{caption}
\usepackage{multirow}
\usepackage{graphicx}
\newtheorem{Key points}{Key points}
\newtheorem{Summary}{Summary}
\usepackage{dblfloatfix}
%\usepackage{adjustbox}
%\usepackage{animate}
%\usepackage{movie15}
%\usepackage{subfig}
%\newtheorem{Definition}{Definition}
%\usepackage[font={small}]{caption}
\usepackage{beamerthemeshadow}
\newcommand\Fontvi{\fontsize{5}{6.2}\selectfont}
\newcommand\Fontvia{\fontsize{6}{7.2}\selectfont}
\newcommand\Fontviaa{\fontsize{8}{7.2}\selectfont}
\usepackage{listings}
\lstset{language=C++,
                keywordstyle=\color{blue},
                stringstyle=\color{red},
                commentstyle=\color{green},
                morecomment=[l][\color{magenta}]{\#},
                numbers=left,
                escapeinside=||
}

%\captionsetup{font=scriptsize,labelfont=scriptsize}
 \usetheme{Antibes}%PaloAlto
\begin{document}
\title[Lecture 8]{Data Structures and Object Oriented Programming using C++} 
\author[]{Ahsan Ijaz}
\date{}
 \frame{\titlepage}
% \AtBeginSection[]
% {
% \begin{frame}<beamer>{Table of Contents}
% \tableofcontents[currentsection,currentsubsection, 
%     hideothersubsections, 
%     sectionstyle=show/shaded,
% ]
% \end{frame}
% }

\section{Polymorphism}
\frame{\frametitle{Polymorphism}
  \begin{itemize}
  \item Polymorphism is a generic term that means 'many shapes'. In C++ the simplest form of Polymorphism is {\color{blue}overloading} of functions.
  \item in the OOP form of polymorphism, a virtual function in the base class Point and then overriding it in the derived class Circle.
  \end{itemize}
 }
\subsection{Pointer to Base Class}
\begin{frame}[fragile]
\frametitle{Pointer to Base Class}
  \begin{definition}
A pointer to a derived class is type-compatible with a pointer to its base class.
  \end{definition}
\begin{lstlisting}
BaseClass BaseObj;
BaseClass * PBClass = &BaseObj;
DerivedClass DerivedObj;
BaseClass * pDclass = &DerivedObj;
\end{lstlisting}
\end{frame}

\begin{frame}[fragile]
\frametitle{Pointer to Base Class}
\Fontviaa
\begin{lstlisting}
#include <iostream>
using namespace std;

class animal
{
public:
void eat();
void sleep();
void breathe();   
};

class fish : public animal
{
public:
void breathe(); //same function as derived class
}

void animal::breathe(){cout<<"Breathing..."<<endl;}
void fish::breathe(){cout<<"Bubbling..."<<endl;}
\end{lstlisting}
\end{frame}

\begin{frame}[fragile]
\frametitle{Main Implementation}
\begin{columns}[onlytextwidth]
    \begin{column}{0.6\textwidth}
      \centering
\begin{lstlisting}
int main()
{
animal animal_object;
fish bigfish;
animal *pointer; //pointer of kind animal

pointer= &animal_object;
pointer->breathe();
/*pointer of kind animal
 accessing a fish object*/
pointer= &bigfish; 
pointer->breathe();

return 0;
}
\end{lstlisting}
\end{column}
\pause    \begin{column}{0.3\textwidth}
      \centering
\textbf{{\color{blue}Output:}} \\Breathing...\\Breathing...\\ 
    \end{column}
\end{columns}
\end{frame}
\begin{frame}[fragile]
\frametitle{Use of Virtual function}
\Fontviaa
\begin{lstlisting}
#include <iostream>
using namespace std;

class animal
{
public:
void eat();
void sleep();
virtual void breathe(); //virtual keyword added
};

class fish : public animal
{
public:
void breathe();
}

void animal::breathe(){cout<<"Breathing..."<<endl;}
void fish::breathe(){cout<<"Bubbling..."<<endl;}
\end{lstlisting}
\end{frame}

\begin{frame}[fragile]
\frametitle{Main Implementation}
\begin{columns}[onlytextwidth]
    \begin{column}{0.6\textwidth}
      \centering
\begin{lstlisting}
int main()
{
animal animal_object;
fish bigfish;
animal *pointer;

pointer= &animal_object;
pointer->breathe();

pointer= &bigfish;
pointer->breathe(); //Now this calls
// function of fish class

return 0;
}
\end{lstlisting}
\end{column}
\pause    \begin{column}{0.3\textwidth}
      \centering
\textbf{{\color{blue}Output:}} \\Breathing...\\Bubbling...\\ 
    \end{column}
\end{columns}

\end{frame}

\subsection{Function Over ride}
\frame{\frametitle{Function Over Ride}
  \begin{definition}
Changing implementation of a function from base class when the function is called through a derived class object.
  \end{definition}
 }

\begin{frame}[fragile]
\frametitle{Function Over Riding Example}
\begin{lstlisting}
class B_class{
	public:
		void functionA(){ /*...*/ }
	};
class D_class : public B_class{
	public:
		void functionA(){ /*...*/ } 
	};
int main(){
	D_class x;
	x.functionA(); 
	}
\end{lstlisting}
\end{frame}
\subsection{}
\begin{frame}[fragile]
\frametitle{Polymorphism-Base Class}
In case of pure virtual function, the base class {\color{blue}\textbf{cannot}} be instantiated or created.
\begin{lstlisting}
class bird
{
public:
  virtual void fly() = 0; // Pure virtual Function
};
\end{lstlisting}
\end{frame}
\begin{frame}[fragile]
\frametitle{Polymorphism-Derived Classes}
\Fontviaa
\begin{lstlisting}
class hummingbird : public bird
{
public:
  virtual void fly()
  {
    flap_wings_like_crazy();
  }
};

class albatross : public bird
{
public:
  virtual void fly()
  {
    glide_majestically_over_the_waves();
  }
};
\end{lstlisting}
\end{frame}
\begin{frame}[fragile]
\frametitle{Polymorphism-Implementation}
Since \textbf{fly()} is virtual function, \textbf{migrate()} will call the appropriate \textbf{fly()} function.
\begin{lstlisting}
void migrate(bird* bird_pointer)
{
   bird_pointer->fly();
}
int main()
{
hummingbird h;
albatross a;

migrate(&h);  
migrate(&a);
}
\end{lstlisting}
\end{frame}
\end{document}