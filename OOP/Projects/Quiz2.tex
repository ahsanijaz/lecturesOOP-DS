\documentclass[11pt]{article}

\makeatletter
\newif\if@restonecol
\makeatother
\let\algorithm\relax
\let\endalgorithm\relax
\let\copyrightbox\relax

%\usepackage{times}
\usepackage{multirow, rotating, amsmath, wasysym, mathrsfs}
\usepackage{subfigure}
\usepackage{algorithm}
\usepackage{algorithmic}
\usepackage{epstopdf}
\usepackage[retainorgcmds]{IEEEtrantools}

% \setlength{\textheight}{22.94cm} % 9in + 0.08cm = 2.5cm margin
% \setlength{\textwidth}{16.59cm}  % 6.5in + 0.08 = 2.5cm margin
% \setlength{\topmargin}{-0.04cm} % for 2.5cm margin
% \setlength{\oddsidemargin}{-0.04cm}   % for 2.5cm margin
% \setlength{\headheight}{0.0in}

% \setlength{\headsep}{0.0in} \setlength{\parindent}{1pc}
% \setlength{\parskip}{0pt}

\def\floatpagefraction{.95}

\title{\textbf{Project 1}\\ Object Oriented Programming\\ and Data Structures \\\textbf{Deadline:} November 4, 2013 }
\date{}

\begin{document}
\maketitle
\newpage

\section*{Project Description}
The goal of this first project is to use object oriented programming
in solving real world problems. The students are expected to
familiarize themselves with open source coding community, libraries
and APIs. \\ A group comprising of two to three should be formed for each project.
 Please note that this project should serve as a
base for the completion of project 2.

\subsection*{Presentation}
Each group will give a 10 minute presentation about the project. This
presentation should cover the API used and discuss it's
functionality. The presentation should also briefly touch the expected outcome of project 2.

\subsection*{Marks Distribution}
Project using open source APIs/library:
\begin{itemize}
\item Presentation (35 marks)
  \begin{itemize}
  \item Clarity of Presentation (10 marks)
  \item Presentation style, timing, speaking skills  (10 marks)
  \item Question answer session (15 marks)
  \end{itemize}
\item API configuration (30 marks) (Installation of library, using library to perform some very basic operation)
\item Illustration of API usage (40)
  \begin{itemize}
  \item Proper comments, proper tabs, ... (10 marks)
  \item Using a basic Class that utilizes functionality of that API (15 marks)
  \item Number of functions used to illustrate API/library (15 marks)
  \end{itemize}
\end{itemize}
Alternatively, if you wish to code the project from scratch (language used for alternative task is limited to C++), the mark distribution would be as follows:
\begin{itemize}
\item Presentation (20 marks)
\item Complexity of task achieved (40 marks)
\item Coding style (20 marks) 
\item Viva (20) 
\end{itemize}

\subsection*{Suggested APIs}
Please note that you are free to use ``any kind'' of
API/library/language. You can develop an android application, write a
twitter data scraper, use image processing library, create a
google/facebook app or use a gaming library to create your very own
game. A list of some good and easy to use APIs using C++ language are
given below, we will also have a demonstration of some of these APIs
in class. Please do not fret about second project yet, just pick up an
API you find interesting. You'll get ideas of Project 2 while working
on them.
\begin{itemize}
\item SDL (Gaming) (Maximum of 4 groups)
\item Allegro (Gaming) (Maximum of 5 groups)
\item SFML (Gaming) (Maximum of 4 groups)
\item openCV (image processing library \textbf{(highly recommended)}) (Maximum of eight groups)
\item Twitter API set~(TwitCurl) (a web programming interface to use twitter
  data) (Maximum 2 groups) 
\item OpenGL and DirectX (advanced gaming APIs, hardware acceleration,
  use of video cards, many other features) (Advance 2D and 3D rendering Libraries)
\item Windows API set (basic interface for developing windows
  application)
\item Many more-Google search your word of interest~(website
  development API, gaming API, Facebook API, Google API, ...) and
  you'll get a plethora of relevant APIs
\end{itemize}

Collaborative work is encouraged in using these APIs :).
\subsection*{Coding from scratch}
If anyone is taking this path, please discuss the project with me before submitting the proposal.

\section*{What's next??}
Please form a group as soon as possible, select a project for
completion and submit the proposal by \textbf{10:40am,~Friday,~October~4,~2013}. 10 marks would be deducted on late submission of proposal.

\section{Honor Code}
It is the responsibility of each group member to participate equally for the completion of task. The question/answer, viva and code checking would be exhaustive for each group. Serious actions would be taken against anyone found in breach of the honor code. (copying project, etc, not knowing what's happening in code (other than what's happening at library level). 
\end{document}

