\documentclass[12pt,addpoints]{exam}
\usepackage{listings}  
\usepackage{booktabs}
\usepackage{multirow}
\usepackage{soul}
\usepackage[normalem]{ulem}
%\usepackage{multicol}
\usepackage{vwcol}
%\usepackage[margin={1cm,1cm}]{geometry}
\lstset{language=C++,numbers=left}
 \pagestyle{headandfoot}
%\extraheadheight[.25in]{}
%\lhead[]{Data Structures \& OOP EC-204 (Continued)}
\chead[\textbf{\uline{National University of Sciences and Technology}}\\
\textbf{\uline{College Of E\&ME}}\\
\textbf{\uline{$1^{st}$ Sessional, Fall 2013}}\\
\textbf{\uline{DE-34 MTS}}]
{}
\rhead[]{Fall, 2013}
\lfoot{}
\cfoot{Page \thepage\ of \numpages}
\rfoot{\iflastpage{End of exam.}{Please go on to the next page\ldots}}
\begin{document}

\begin{center}
\begin{vwcol}[widths={0.5,0.5},rule=none,sidesep=5pt]
\phantom{dss}Subject Code: EC 204\\
Date: October 22, 2013\\
Total Marks: 50\\
Subject: Data Structures \& OOP\\
Timing: 9:50am-10:50am\\
Time: 1 hour 
\end{vwcol}
\end{center}
\vspace{0.1in}

\makebox[\textwidth]{Name :\enspace\hrulefill{}~~Section:\enspace\hrulefill}

\vspace{0.2in}

\makebox[\textwidth]{Registration Number:\enspace\hrulefill}

\vspace{0.2in}

Instructor: Ahsan Ijaz\\
\noindent\makebox[\linewidth]{\rule{\textwidth}{1pt}}
\vspace{0.2in}
\begin{questions}
\question
Fill in the following blanks:
\begin{parts}
\part[1]
Constructors are called when objects are \underline{\phantom{something}}.\\
\part[1]
Destructors are used to \underline{\phantom{something somee}}.\\
\part[1]
Every Object can access its own members using \underline{\phantom{something}} pointer.\\
\part[2]
In single inheritance, constructor of \underline{\phantom{something}} class is called before the call of \underline{\phantom{something}} class constructor.  
%\vspace{0.2in}
\end{parts}
\vspace{0.3in} 
\question
Indicate as true or false. In case of false, suggest correction:
\begin{parts}
\part[1]
Constructor can be treated as a public member function and can therefore have any return type.\\
\begin{oneparcheckboxes}
\choice True
\choice False
\end{oneparcheckboxes}\\
\part[1]
In Public inheritance, protected members can be accessed in derived class.\\
\begin{oneparcheckboxes}
\choice True
\choice False
\end{oneparcheckboxes}\\
\part[1]
Two global functions, with the same name, same arguments and same return type can be called in main by defining them in different places.\\
\begin{oneparcheckboxes}
\choice True
\choice False
\end{oneparcheckboxes}\\
\part[1]
Friend functions cannot access protected members.\\
\begin{oneparcheckboxes}
\choice True
\choice False
\end{oneparcheckboxes}\\
\part[1]
Destructors need to be explicitly called before going out of scope.\\
\begin{oneparcheckboxes}
\choice True
\choice False
\end{oneparcheckboxes}\\
\end{parts}
\vspace{0.2in} 
\newpage
\question[15]
In the table below, cross out the wrongly placed ``yes''. As an example, private members in public inheritance cannot be accessed in derived class and is hence crossed out in the table.
\vspace{\fill}
%\vspace{\stretch{1}}
\begin{table}[htpb] \centering
 \begin{tabular}{lcccc} \toprule
 Accessibility from  & Inheritance Type & Public  & Protected & Private\\\midrule 
\multirow{3}{*}{Derived Class} & Public & Yes & Yes & \textbf{\st{Yes}}\\
& Private & Yes & Yes & Yes\\
& Protected & Yes & Yes & Yes\\
\midrule 
% \multirow{3}{*}{Class Derived from derived class} & Public & Yes & Yes & Yes\\
% & Private & Yes & Yes & Yes\\
% & Protected & Yes & Yes & Yes\\
% \midrule 
\multirow{3}{*}{Main Function} & Public & Yes & Yes & Yes\\
& Private & Yes & Yes & Yes\\
& Protected & Yes & Yes & Yes\\
 \bottomrule
 \end{tabular}
 \caption{Member access in Inheritance}
 \end{table}
\vspace{\fill}
\question
Refer to the following code snippet for answering the following question parts:
\begin{lstlisting}
#include <iostream>
using namespace std;
class Furniture
{
   public:
      Furniture( int len );
      Furniture( const Furniture &obj);
      Furniture();
      ~Furniture();               
  private:
      int *ptr;
};

Furniture::Furniture(int len)
{
    ptr=new int;
    *ptr = len;
}

Furniture::Furniture()
{
}

Furniture::Furniture(const Furniture &obj)
{
    ptr = new int;
   *ptr = *obj.ptr; 
}
Furniture::~Furniture(void)
{
    delete ptr;
}
\end{lstlisting}
\begin{parts}
\part[3] Identify constructor, copy constructor, constructor overloading, destructor and member access specifiers from the given code snippet (just write line number,~e.g.~Class Definition: Line\#~3)
\vspace{0.1in}
%\fillwithlines{2in}
\part[5]
Error would occur in case of object instantiation~(e.g.~Furniture bed;) using the above code, identify the source and type of error.
\vspace{0.1in}
%\fillwithlines{2in}
%\newpage
\part[2] What is the advantage of using \textbf{\&} operator in the copy constructor. Talk in terms of memory allocation.
\vspace{0.1in}
%\fillwithlines{1.5in}
\end{parts}

%  \question
% Short Questions:
% \begin{parts}
%  \part[5]
% Write down the two methods of writing a constructor of the following class:
%  \begin{lstlisting}
% class Animal
% {
% public:
% int legs;
% }
%  \end{lstlisting}
% \fillwithlines{\stretch{1}}
%  \part[5]
% Explain the use of scope resolution operator and it's usage in namespaces.~(Just explanation required, no need for writing a code.)
%  \vspace{\stretch{1}}


% \end{parts}
% Please answer briefly to the following questions, use the space provided:
% \begin{parts}
% \part[5]
% Describe the effect on the balloon industry.
% \vspace{\stretch{1}}
% \part[5]
% Describe the effect on the aircraft industry.
% \vspace{\stretch{1}}
% \end{parts}
% \question[20]
% \begin{parts}
% \part
% Define the universe. Give three examples.
% \fillwithlines{\stretch{1}}
% \part
% If the universe were to end, how would you know?
% \end{parts}
\question
Write a code that:
\begin{parts}
\part[2] Has a class named \textbf{CoolingDevices} with public members as \textbf{Type} and  \textbf{price}.
\part[3] Has a class inherited from \textbf{CoolingDevices} named \textbf{AirConditioner} with public members \textbf{price}, \textbf{PowerConsumption}, \textbf{CoolingCapacity} and a private member \textbf{thermostat}.
\part[5] Overload operator \textbf{``+''} so that by adding two objects of type \textbf{AirConditioner}, there public members \textbf{PowerConsumption} and \textbf{CoolingCapacity} are added together.
%\part Overload operator \textbf{``++''} of the derived class so that the private member \textbf{Thermostat} is incremented when the operator is called.
\part[5] Both the classes have the same member \textbf{price}. Briefly discuss scope of the variable price and how either one of it can be accessed in the derived class.
\vspace{0.1in}
%\fillwithlines{\stretch{1}}
\end{parts}
\end{questions}
\begin{center}
This exam has \numquestions\ questions, for a total of \numpoints\
points.
\end{center}
\begin{center}
\gradetable[h][questions]
\end{center}
%\vspace{0.5in}
%\vspace*{\fill}
\end{document}