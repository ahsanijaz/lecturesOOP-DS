\documentclass[11pt]{article}

\makeatletter
\newif\if@restonecol
\makeatother
\let\algorithm\relax
\let\endalgorithm\relax
\let\copyrightbox\relax

%\usepackage{times}
\usepackage{multirow, rotating, amsmath, wasysym, mathrsfs}
\usepackage{subfigure}
\usepackage{algorithm}
\usepackage{algorithmic}
\usepackage{epstopdf}
\usepackage[retainorgcmds]{IEEEtrantools}

\setlength{\textheight}{22.94cm} % 9in + 0.08cm = 2.5cm margin
\setlength{\textwidth}{16.59cm}  % 6.5in + 0.08 = 2.5cm margin
\setlength{\topmargin}{-0.04cm} % for 2.5cm margin
\setlength{\oddsidemargin}{-0.04cm}   % for 2.5cm margin
\setlength{\headheight}{0.0in}

\setlength{\headsep}{0.0in} \setlength{\parindent}{1pc}
\setlength{\parskip}{0pt}

\def\floatpagefraction{.95}

\title{\textbf{Lab Report 3}\\ Object Oriented Programming\\ and Data Structures }

\date{}

\begin{document}
\maketitle
\begin{enumerate}
\item Create a class named 'Friend' with private data members (age, name,
height,$\cdots$,etc).  
\item  The class should have at least one private data
member which is declared using dynamic memory allocation, \emph{\textbf{also create a private data member of type int ``rank''.}}
\item  Write a constructor
that can be used to initialize the values of private data
members. 
\item  Overload the constructor so that instantiation of an object
is also possible. 
\item Write a copy constructor as well. 
\item Write a public member function animate() that displays \emph{``In animate routine''} and request user to enter an animation request from a menu.
e.g. 1. Run(), 2. Jump(), 3. Eat(),..., or invalid option.
\item Create Private member functions Run(), Jump(), Eat() that are called from function animate() written as a public function. The functions should simply display something like \emph{``in Run() routine''}. 
\item Write destructor for the class that deletes the dynamically allocated element while displaying \emph{``In destructor routine of Object \textbf{Name}''}. 
\item Illustrate usage of this class in the main routine of the program.
\item \emph{\textbf{Overload operator '++' that increments the 'rank' of the friend, write ++ operator in both postfix and prefix format. Use 'this pointer' while referring to the 'rank' data member in the overload function defintion.}}
\end{enumerate}

\end{document}
