\documentclass{beamer}
\setbeamercovered{transparent}
\usepackage{epstopdf}
\usepackage{listings}
\usepackage{lipsum}
\usepackage{subfig}
\usepackage{algorithm}
\usepackage{algorithmicx}
\usepackage{cite}
\usepackage{lipsum}
\usepackage{amssymb}
\usepackage{color}
\usepackage{IEEEtrantools}
\usepackage{booktabs}
\usepackage{texpower}
\usepackage{amsmath}
\usepackage{caption}
\usepackage{multirow}
\usepackage{graphicx}
\newtheorem{Key points}{Key points}
\newtheorem{Summary}{Summary}
\usepackage{dblfloatfix}
%\usepackage{adjustbox}
%\usepackage{animate}
%\usepackage{movie15}
%\usepackage{subfig}
%\newtheorem{Definition}{Definition}
%\usepackage[font={small}]{caption}
\usepackage{beamerthemeshadow}
\newcommand\Fontvi{\fontsize{5}{6.2}\selectfont}
\newcommand\Fontvia{\fontsize{6}{7.2}\selectfont}
\newcommand\Fontviaa{\fontsize{8}{7.2}\selectfont}
\usepackage{listings}
\lstset{language=C++,
                keywordstyle=\color{blue},
                stringstyle=\color{red},
                commentstyle=\color{green},
                morecomment=[l][\color{magenta}]{\#},
                numbers=left,
                escapeinside=||
}

%\captionsetup{font=scriptsize,labelfont=scriptsize}
 \usetheme{Antibes}%PaloAlto
\begin{document}
\title[Lecture 2]{Data Structures and Object Oriented Programming using C++} 
\author[]{Ahsan Ijaz}
\date{}
 \frame{\titlepage}
% \AtBeginSection[]
% {
% \begin{frame}<beamer>{Table of Contents}
% \tableofcontents[currentsection,currentsubsection, 
%     hideothersubsections, 
%     sectionstyle=show/shaded,
% ]
% \end{frame}
% }
\section{Review of Last Lecture}
\frame{\frametitle{Topics Covered}
  \begin{itemize}
  \item<1-> Classes
  \item<2-> Objects
  \item<3-> Data Hiding
  \end{itemize}
}
\section{}
\frame{\frametitle{Lecture Details}
  \begin{itemize}
  \item Constructors
  \item Destructors
  \end{itemize}
}
\section{Constructors}
\frame{\frametitle{Constructors}
  \begin{itemize}
  \item Member Function 
  \item \textbf{Primary use}-Object Initialization
   \end{itemize}
}

\begin{frame}[fragile]
\frametitle{Example Class}
\Fontvia
%\begin{small}
\begin{lstlisting}
class Distance
{private:
	int feet;
	float inches;
 public:
	 void setdistance(int ft, float in)  //Function Used to Initialize Object
		{feet = ft;
		 inches = in;}
	 void getdistance()
		{cout << "Enter feet : "; 
		 cin >> feet;
		 cout << "Enter inches : "; 
		 cin >> inches;}
	 void showdistance()
		{cout << "feet"<< feet << "inches" << inches << endl;}
};
int main()
{	Distance d1, d2;
 	d1.setdistance(1,2);
	d1.showdistance();
	d2.getdistance();
	d2.showdistance();
}
\end{lstlisting}
\end{frame}

\frame{\frametitle{Constructors Definition}
  \begin{itemize}
  \item<1-> Name Should be same as Class
  \item<2-> No Return Type
  \item<3-> Function called at time of Object creation
  \item<4-> Increases program readability  
  \end{itemize}
}
\begin{frame}[fragile]
\frametitle{Writing Constructors}
\begin{itemize}
\item<1> Counter() {count=0;}
\item<2-> Counter(): count(0) \{\}; 
\end{itemize}
\end{frame}
\begin{frame}[fragile]
\frametitle{Constructors Example 1}
\Fontvi
\begin{lstlisting}
class Counter
{
 private:
	unsigned int count;
 public:
	Counter() {count=0;}    //CONSTRUCTOR DEFINITION
	void inc_count()
		{count++;}  // increment counter
	int get_count()
		{return count;}  // return counter
};

int main()
{
	Counter c1, c2;
	
	 cout << "\n c1 =" << c1.get_count();    // display counter 1
	 cout << "\n c2 =" << c2.get_count();    // display counter 2 

	 c1.inc_count();    // increment counter 1
	 c2.inc_count();    // increment counter 2
	 c2.inc_count();    // increment counter 2

	 cout << "\n c1 =" << c1.get_count();    // display again
	 cout << "\n c2 =" << c2.get_count() << endl;
}
\end{lstlisting}
\end{frame}
\begin{frame}[fragile]
\frametitle{Constructors Example 2}
\Fontvi
\begin{lstlisting}
class Distance
{private:
	int feet;
	float inches;
 public:
	 Distance(int ft, float in):feet(ft),inches(in) 
		{};  //CONSTRUCTOR DEFINITION
	 void getdistance()
		{cout << "Enter feet : "; 
		 cin >> feet;
		 cout << "Enter inches : "; 
		 cin >> inches;}
	 void showdistance()
		{cout << feet << "\' - " << inches << "\" "<< endl;}
};

int main()
{	Distance d1(11,6.25);
 	d1.showdistance();
	d1.getdistance();
	d1.showdistance();
}
\end{lstlisting}
\end{frame}
 \frame{\frametitle{Constructor Overloading}
   \begin{itemize}
   \item Initialize one of the objects, instantiate another one.
   \item Use multiple constructors
   \item Similar as Function Overloading 
   \end{itemize}
}
\begin{frame}[fragile]
\frametitle{Constructor Overloading Example}
\Fontviaa
\begin{lstlisting}
class Distance
{private:
  int feet;
  float inches;
public:
  Distance()   //CONSTRUCTOR 1 
      {}
     Distance(int ft, float in): feet(ft),inches (in){}; 
  …. //CONSTRUCTOR 2
};

int main()
{
 	Distance dist1, dist3; 
	Distance dist2(11, 6.25); 
	 …..
}	
\end{lstlisting}
\end{frame}

\section{Destructor}
 \frame{\frametitle{Destructor}
   \begin{itemize}
   \item<1-> Opposite of Constructors
   \item<2-> Destructors should be called before object goes out of scope
   \item<3-> No arguments, no return types
   \item<4-> De-allocation of memory 
   \end{itemize}
}
\begin{frame}[fragile]
\frametitle{Destructor Example}
\Fontviaa
\begin{lstlisting}
class Counter
{private:
	int *data;
 public:
	Counter():count (0) {}
	~Counter() 
		{delete data;}
};
\end{lstlisting}
\end{frame}
\begin{frame}[fragile]
\frametitle{Copy Constructor}
\begin{itemize}
\item Initialize one object from another of the same type.
\item Created by default
\end{itemize}
\end{frame}
\begin{frame}[fragile]
\frametitle{Complete Example for a Class}
\Fontvi
\begin{lstlisting}
#include <iostream>
using namespace std;
class Line
{
   public:
      Line( int len );             // simple constructor
      Line( const Line &obj);  // copy constructor
      ~Line();                     // destructor
  private:
      int *ptr;
};
// Member functions definitions including constructor
Line::Line(int len)
{
    cout << "Normal constructor allocating ptr" << endl;
    // allocate memory for the pointer;
    ptr = new int;
    *ptr = len;
}
Line::Line(const Line &obj)
{
    cout << "Copy constructor allocating ptr." << endl;
    ptr = new int;
   *ptr = *obj.ptr; // copy the value
}
Line::~Line(void)
{
    cout << "Freeing memory!" << endl;
    delete ptr;
}

\end{lstlisting}
\end{frame}
\begin{frame}[fragile]
\frametitle{Namespaces}
\begin{itemize}
\item Resolves Conflicts while using functions of variables defined at different places but having the same name.
\end{itemize}
\Fontvi
\begin{lstlisting}
#include <iostream>
using namespace std;

// first name space
namespace first_space{
   void func(){
      cout << "Inside first_space" << endl;
   }
}
// second name space
namespace second_space{
   void func(){
      cout << "Inside second_space" << endl;
   }
}
int main ()
{
 
   // Calls function from first name space.
   first_space::func();
   
   // Calls function from second name space.
   second_space::func(); 

   return 0;
}
\end{lstlisting}
\end{frame}
\end{document}